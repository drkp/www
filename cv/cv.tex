% -*- mode: latex; TeX-PDF-mode:t; -*-
\documentclass[10pt,times]{report}
\usepackage[letterpaper,text={6.5in,9.5in},centering,nofoot]{geometry}
% use the showframe option in the above line to show the frame
\usepackage{microtype}


\setlength{\tabcolsep}{0pt}
\setlength{\parindent}{0pt}
\setlength{\parsep}{0pt}


% Separation between header and body
\newlength{\partgap}
\setlength{\partgap}{.2in}

% Additional separation between sections
\newlength{\sectiongap}
\setlength{\sectiongap}{.6em}

% Separation between entries
\newlength{\entrygap}
\setlength{\entrygap}{.25em}

\newlength{\sectioncolwidth}
\setlength{\sectioncolwidth}{1in}
\newlength{\colgap}
\setlength{\colgap}{.5em}
\newlength{\stuffwidth}
\setlength{\stuffwidth}{\textwidth}
\addtolength{\stuffwidth}{-\colgap}
\addtolength{\stuffwidth}{-\sectioncolwidth}
\addtolength{\stuffwidth}{-.5em}  % minipage margins?

\pagestyle{empty}


% From TeX by Topic
\def\ifEqString#1#2{\def\testa{#1}\def\testb{#2}%
  \ifx\testa\testb}

\newenvironment{rtable}{
  \begin{minipage}{\textwidth}
  }{
  \end{minipage}
}

\newenvironment{rentry}[3][xxx]{
  \begin{minipage}[t]{\hsize}
    \textbf{#2}\ifEqString{#1}{xxx}\relax\else, \textit{#1}\fi
    \hspace{\stretch{1}} #3 \\
  }{
    \removelastskip
  \end{minipage}
  \\[\entrygap]  % Useful for page squeezing
}

\newcommand{\rline}[2]{
  \begin{minipage}[t]{\hsize}
    #1 \hspace{\stretch{1}} #2
  \end{minipage} \\
}

\newenvironment{rsection}[1]{
  \begin{minipage}[t]{\sectioncolwidth}
    \textsc{#1}
  \end{minipage}
  \hspace{\colgap}
  \begin{minipage}[t]{\stuffwidth}
  }{
    \removelastskip
  \end{minipage}
  \\[\sectiongap]
}

\newenvironment{ritemize}{%
  \begin{list}{$\cdot$}{\topsep 0pt \parskip 0pt \partopsep 0pt
      \itemsep 0pt \parsep 0pt}%
}{\end{list}}

%%%%%%%%%%%%%%%%%%%%%%%%%%%%%%%%%%%%%%%%%%%%%%%%%%%%%%%%%%%%%%%%%%

\begin{document}

% Name
\begin{center}
  \LARGE{\sc{Irene Y. Zhang}}
\end{center}
\vspace{2mm}

% Contact info
\begin{tabular*}{\textwidth}{l@{\extracolsep{\fill}}r}
  185 NE Stevens Way & \texttt{iyzhang@cs.washington.edu} \\
  Seattle, WA  98195 & \texttt{http://irenezhang.net} \\ 
\end{tabular*}

\vspace{\partgap}

% Table
\begin{rtable}
  \begin{rsection}{Education}
    \begin{rentry}{University of Washington}{Seattle, WA}
      \rline{Ph.D. in Computer Science and Engineering}{}
      Advisors: Hank Levy and Arvind Krishnamurthy
    \end{rentry}

    \begin{rentry}{University of Washington}{Seattle, WA}
      \rline{M.S. in Computer Science and Engineering}{December 2013}
      Advisors: Hank Levy, Arvind Krishnamurthy, and Steve Gribble\\
      Thesis: \textit{Simplifying Mobile/Cloud Applications with Sapphire}
    \end{rentry}

    \begin{rentry}{Massachusetts Institute of Technology}{Cambridge,
        MA} \rline{M.Eng. in Electrical Engineering and Computer
        Science}{June 2009} Advisor: M. Frans Kaashoek\\
      Thesis: \textit{Efficient File Distribution in a Flexible, Wide-area
        File System}%\\Graduate GPA: 5.0/5.0
    \end{rentry}

    \begin{rentry}{Massachusetts Institute of Technology}{Cambridge, MA}        
        \rline{S.B. in Computer Science and Engineering}{June 2008}
        %Undergraduate GPA: 4.7/5.0
    \end{rentry}
  \end{rsection}

  \begin{rsection}{Interests}
    Operating systems, distributed systems, virtualization and networking
  \end{rsection}
  
  \begin{rsection}{Research}
    \begin{rentry}{Building Consistent Transactions with Inconsistent
        Replication}{}
      TAPIR -- the Transactional Application Protocol for Inconsistent
      Replication -- provides externally consistent transactions using a
      replication protocol with \emph{no consistency guarantees}.
      Unlike conventional protocols that use Paxos, TAPIR does not
      require a Paxos leader or coordination between replicas in a
      shard. Thus, TAPIR can commit a transaction \emph{in a single
        round-trip} and eliminate the bottleneck at the Paxos leader.
    \end{rentry}

    \begin{rentry}{Customizable and Extensible Deployment for
        Mobile/Cloud Applications}{}
      Sapphire is a new distributed programming platform providing
      customizable and extensible deployment of mobile/cloud
      applications. The key concept is an architecture that supports
      \emph{deployment managers}, which solve complex distributed
      systems tasks, such as code-offloading and caching. Rather than
      writing distributed systems code, programmers compose a custom
      deployment to meet their application's needs.
    \end{rentry}

    \begin{rentry}{User-controlled Privacy for Mobile/Cloud Applications}{}
      Agate is a new trusted distributed runtime system that gives
      users control over how mobile/cloud applications share sensitive
      user data collected on mobile devices (e.g., photos, GPS
      location). Agate combines aspects of access control and
      information flow control to allow applications to share user
      data in application-specific ways, while enforcing user policies
      without trusting the application or the application programmer.
    \end{rentry}

    \begin{rentry}{Arrakis: The Operating System is the Control Plane}{}
      Arrakis is a new operating system that provides high performance
      I/O by taking advantage of hardware virtualization
      technology. Hardware virtualization technologies are designed to
      eliminate the hypervisor from fast-path I/O operations. Arrakis
      takes this technology a step further by using it to eliminate
      the operating system as well, allowing applications to directly
      access the hardware during normal execution and providing
      significantly better performance, reliability and
      customizability.
    \end{rentry}

    \begin{rentry}{Improving VM Checkpoint Restore Performance}{} 
      With collaborators at VMware, I developed two techniques for
      improving the performance of restoring checkpointed virtual
      machines. The first estimates and prefetches the working set of
      the checkpointed VM on restore, improving the responsiveness of
      the VM during restore. The second groups memory pages together
      on disk that are likely to be accessed together, improving disk
      efficiency during restore.
    \end{rentry}

    % \begin{rentry}{Transactional Consistency and Automatic Cache Management}{}
    %   In-memory application caches like memcached introduce
    %   significant complexity to web applications because they provide
    %   a simple get-put interface, violate the consistency guarantees
    %   of the underlying database and leave cache invalidations to the
    %   application. TxCache addresses these problems by providing a
    %   programming model for caching, transactional consistency across
    %   the entire system and automatic cache invalidations generated by
    %   the database.
    % \end{rentry}

    % \begin{rentry}{WheelFS and Large File Distribution}{}
    %   The goal of WheelFS is to provide a general wide-area storage
    %   solution with a standard POSIX interface. The challenge is that
    %   any storage system operating in the wide-area must make
    %   trade-offs for performance and trade-off decisions are often
    %   better made by applications. WheelFS solves this problem by
    %   allowing applications to configure the file system using
    %   keywords embedded in the pathname. My thesis work focused on
    %   adding support for efficient file distribution to WheelFS for
    %   distributing large files and handling flash crowds.
    % \end{rentry}

  %\vspace{1.0em}
%   \begin{rsection}{}
%     \begin{rentry}{Fresh Breeze Multiprocessor Architecture}{}
%       Work with Prof. Jack Dennis implementing a cycle-accurate
%       simulator for the Fresh Breeze multiprocessor architecture.  Led
%       a 3 student team in the design and implementation of a tool for
%       interacting with the Fresh Breeze simulator. The goal of the
%       Fresh Breeze architecture is to provide a sound base for
%       executing programs written following the principles of modular
%       software construction. The architecture introduces three major
%       departures from conventional multiprocessor design: simultaneous
%       multithreading, global shared address space, and no memory
%       update, cycle-free heap.
%     \end{rentry}
%  \end{rsection}
        
  \end{rsection}
\end{rtable}

\begin{rtable}
 \begin{rsection}{Conference\\Publications}
   \textbf{Irene Zhang}, Naveen Kr. Sharma, Adriana Szekeres, Dan
   R. K. Ports, Arvind Krishnamurthy. \textit{Building Consistent
     Transactions with Inconsistent Replication.}.  In Proceedings of
   the ACM Symposium on Operating Systems Principles (SOSP). Monterey,
   CA. October 2015.\\\vspace{-0.5em}

   \textbf{Irene Zhang}, Adriana Szekeres, Dana Van Aken, Isaac
   Ackerman, Steven D. Gribble, Arvind Krishnamurthy, Henry
   M. Levy. \textit{Customizable and Extensible Deployment for
     Mobile/Cloud Applications.}  In Proceedings of the USENIX Symposium
   on Operating Systems Design and
   Implementation (OSDI). Broomfield, CO. October 2014.\\ \vspace{-0.5em}

   Simon Peter, Jialin Li, \textbf{Irene Zhang}, Dan R. K. Ports, Doug
   Woos, Arvind Krishnamurthy, Thomas Anderson, Timothy Roscoe.
   \textit{Arrakis: The Operating System is the Control Plane.}  In
   Proceedings of the USENIX Symposium on Operating Systems Design and
   Implementation (OSDI).
   Broomfield, CO. October 2014. \textbf{Best Paper
     Award}.\\\vspace{-0.5em}

   \textbf{Irene Zhang}, Tyler Denniston, Yury Baskakov, Alex
   Garthwaite.   \textit{Optimizing VM Checkpointing for Restore
     Performance in VMware ESXi.}
   In Proceedings of the USENIX Annual Technical Conference (USENIX ATC).
   San Jose, CA. June 2013.\\\vspace{-0.5em}

   \textbf{Irene Zhang}, Alex Garthwaite, Yury Baskakov, Kenneth
   C. Barr. \textit{Fast Restore of Checkpointed Memory Using Working
     Set Estimation.}  In Proceedings of the ACM Conference on Virtual
   Execution Environments (VEE). Newport Beach, CA. March 2011.\\\vspace{-0.5em}

   Dan R. K. Ports, Austin Clements, \textbf{Irene Zhang}, Samuel
   Madden, Barbara Liskov. \textit{Transactional Consistency and
     Automatic Management in an Application Data Cache.}  In
   Proceedings of the USENIX Symposium on Operating Systems Design and
   Implementation (OSDI). Vancouver, Canada. October 2010.\\\vspace{-0.5em}

   Jeremy Stribling, Yair Sovran, \textbf{Irene Zhang}, Xavid Pretzer,
   Jinyang Li, M. Frans Kaashoek, Robert Morris. \textit{Flexible,
     Wide-Area Storage for Distributed Systems with WheelFS.}  In
   Proceedings of the USENIX Symposium on Networked Systems Design and
   Implementation (NSDI).  Boston, MA. April 2009. \\\vspace{-0.5em}
  \end{rsection}

  \begin{rsection}{In Submission}    
    Adriana Szekeres, \textbf{Irene Zhang}, Katelin Bailey, Isaac
    Ackerman, Haichen Shen, Franziska Roesner, Dan R. K. Ports, Arvind
    Krishnamurthy, Henry M. Levy. \textit{Operating Systems Support
      for User-controlled Privacy.}\\\vspace{-0.5em}
  \end{rsection}
  
  \begin{rsection}{Workshop Publications}
    Brandon Holt, \textbf{Irene Zhang}, Dan R. K. Ports, Mark Oskin and Luis
    Ceze.  \textit{Claret: Using Data Types for Highly Concurrent
      Distributed Transactions.} In Proceedings of the Workshop on
    Principles and Practice of Consistency for Distributed Data
    (PaPoC).  Bordeaux, France. April 2015.\\\vspace{-0.5em}


    Simon Peter, Jialin Li, Doug Woos, \textbf{Irene Zhang}, Dan
    R. K. Ports, Thomas Anderson, Arvind Krishnamurthy, Mark
    Zbikowski. \textit{Towards High-Performance Application-Level
      Storage Management.} In Proceedings of the
    USENIX Workshop on Hot Topics in Storage and File Systems (HotStorage). Philadelphia, PA. June 2014. \\\vspace{-0.5em}
  \end{rsection}

  \begin{rsection}{Posters \&\\Extended\\Abstracts}
    \textbf{Irene Zhang}, Naveen Kr. Sharma, Adriana Szekeres, Dan
    R. K. Ports, Arvind Krishnamurthy. \textit{Optimistic, Replicated
      Two-Phase Commit.}  ACM Asia-Pacific Workshop on Systems (APSys). Beijing, China. June 2014.\\\vspace{-0.5em}

    \textbf{Irene Zhang}, Alex Garthwaite, Yury Baskakov, Kenneth
    C. Barr, Jesse Pool, Kevin Christopher. \textit{Fast Restore of
      Checkpointed Memory Using Working Set Estimation.} ACM Symposium
    on Operating Systems Principles (SOSP).  Big Sky, MT. October
    2009.\\\vspace{-0.5em}

    \textbf{Irene Zhang}, Kenneth C. Barr.  \textit{Improving VMware
      Workstation Restore using Working Set Estimation.}  VMworld
    Conference. Las Vegas, NV. September 2008.\\ \vspace{-0.5em}
  \end{rsection}
\end{rtable}

\begin{rtable}
  \begin{rsection}{Awards}

    \begin{rentry}{Google Anita Borg Memorial Fellowship}{2015}
      \vspace{-0.5em}
    \end{rentry} 
    \begin{rentry}{Microsoft Research PhD Fellowship}{2015}
      \vspace{-0.5em}
    \end{rentry} 
    \begin{rentry}{Industrial Affiliates Madrona Prize}{2014}
       \vspace{-0.5em}
    \end{rentry}
    \begin{rentry}{OSDI Best Paper Award}{2014}
      \vspace{-0.5em}
    \end{rentry}
    \begin{rentry}{National Science Board Annual Meeting Student
        Panel}{2013}
      \vspace{-0.5em}
    \end{rentry}
    \begin{rentry}{National Science Foundation Fellowship}{2013}
      \vspace{-0.5em}
    \end{rentry}
    \begin{rentry}{ARCS Foundation Fellowship}{2012}
       \vspace{-0.5em}
    \end{rentry}
    \begin{rentry}{Jeff Dean and Heidi Hopper Endowed Regental Fellowship}{2012}
       \vspace{-0.5em}
    \end{rentry}
    \begin{rentry}{VMware Academic Program Top Intern Project}{2008}
       \vspace{-0.5em}
    \end{rentry}
    \begin{rentry}{CRA Outstanding Undergraduate Award, Honorable
      Mention}{2008}
    \vspace{-0.5em}
    \end{rentry}
    \begin{rentry}{Northern Telecom/BNR Award for Best Undergrad. Lab
        Project}{2006}
      \vspace{-0.5em}
    \end{rentry}
  \end{rsection}

  \begin{rsection}{Patents}
    US Patent App. 12/559,484.\\
    \textit{Saving and Restoring State Information for Virtualized Computer Systems.}\\
    \textbf{I. Zhang}, K. C. Barr, G. Venkitachalam, I. Ahmad, A. Garthwaite, J. Pool.\\\vspace{-0.5em}

    US Patent App. 13/710,185.\\
    \textit{Method for Saving Virtual Machine State from a Checkpoint
    File.}\\
    A. Garthwaite, Y. Baskakov, \textbf{I. Zhang}, K. Christopher,
    J. Pool.\\\vspace{-0.5em}

    US Patent App. 13/710,215.\\
    \textit{Method for Restoring Virtual Machine State from a Checkpoint
    File.}\\
    A. Garthwaite, Y. Baskakov, \textbf{I. Zhang}, K. Christopher,
    J. Pool.\\\vspace{-0.5em}
    
    US Patent App. 13/935,382.\\
    \textit{Identification of Page Sharing Opportunities within Large
      Pages.}\\
    Y. Baskakov, A. Garthwaite, R. Venkatasubramanian, \textbf{I. Zhang}, S. Kim, N. Bhatia, K. Tati\\
  \end{rsection}

  \begin{rsection}{Talks}
    \begin{rentry}{Operating Systems for Modern Applications}{}
      \rline{CSE Symposium}{Jan 2015}
      \vspace{-0.5em}
    \end{rentry}
    \begin{rentry}{Building Consistent Transactions with Inconsistent
        Replication}{}
      \rline{Amazon Tech Talk, Host: Andrew Certain}{Nov 2014}
      \vspace{-0.5em}
    \end{rentry}
    \begin{rentry}{Customizable and Extensible Deployment for
        Mobile/Cloud Applications}{}
      \rline{MSR Tech Talk, Host: Phil Bernstein}{Nov 2014}
      \rline{UW CSE Industrial Affiliates Meeting}{Oct 2014}
      \rline{Symposium on Operating Systems Design and
        Implementation (OSDI)}{Oct 2014}
      \rline{UW Systems Seminar}{Oct 2014}
      \rline{Symposium on Operating Systems Principles (SOSP) Work-in-Progress}{Nov 2013}
      \rline{UW/MSR Research Day}{Apr 2013}
      \vspace{-0.5em}
    \end{rentry}
    \begin{rentry}{Optimizing VM Checkpointing for Restore Performance
      in VMware ESXi}{} 
      \rline{USENIX Annual Technical Conference (USENIX ATC)}{Jun
        2013}
      \vspace{-0.5em}
    \end{rentry}
    \begin{rentry}{Fast Restore of Checkpointed Memory using Working
        Set Estimation}{}
      \rline{University of Washington Tech Talk}{Oct 2011}
      \rline{Cornell SWE Tech Talk}{Sep 2011}
      \rline{Conference on Virtual Execution Environments (VEE)}{Mar
        2011}
      \vspace{-0.5em}
    \end{rentry}
  \end{rsection}

  \begin{rsection}{Press}
    \textit{Cutting-edge server operating system wins UW computer
      science prize.} GeekWire. October 23, 2014.\\\vspace{-0.5em}

    \textit{Faster websites, more reliable data.}
    MIT News. October 14, 2010.
  \end{rsection}
\end{rtable}

\begin{rtable}

  \begin{rsection}{Service}
    \begin{rentry}{UW Conference on Potentially
        Computer Science (PoCSci)}{}
      \rline{Program Co-chair}{2015}
       \vspace{-0.5em}
    \end{rentry}
    \begin{rentry}{UW Graduate Student Committee}{}
      \rline{Graduate Women's Event Coordinator}{2014-2015}
      \rline{Graduate Visit Days Committee Co-chair}{2013-2014}
      \vspace{-0.5em}
    \end{rentry}
    \begin{rentry}{UW Graduate Student Mentor}{2013-2014}
      \vspace{-0.5em}
    \end{rentry}
    \begin{rentry}{VMware Women's Outreach and Recruiting}{2009-2012}
       \vspace{-0.5em}
    \end{rentry}
    \begin{rentry}{Eta Kappa Nu EECS Honor Society Officer}{2008-2009}
      \vspace{-0.5em}
    \end{rentry}
  \end{rsection}

  \begin{rsection}{Teaching}
    \begin{rentry}{Distributed Systems (CSE 452)}{}
      \rline{Teaching Assistant, UW Department of CSE}{Winter
        2015}   
      \vspace{-0.5em}   
    \end{rentry}
    \begin{rentry}{Introduction to Operating Systems (CSE 451)}{}
      \rline{Tutor, UW Department of CSE}{Fall 2014}
      \rline{Tutor, UW Department of CSE}{Spring 2014}
      \rline{Guest Lecturer, UW Department of CSE}{Fall 2013}
      \rline{Tutor, UW Department of CSE}{Spring 2013}
      \vspace{-0.5em}
    \end{rentry}
    \begin{rentry}{The Hardware/Software Interface (CSE 351)}{}
      \rline{Tutor, UW Department of CSE}{Winter 2014}
      \rline{Tutor, UW Department of CSE}{Winter 2013}
      \vspace{-0.5em}
    \end{rentry}
    \begin{rentry}{Operating Systems Engineering (6.828)}{}
      \rline{Teaching Assistant, MIT Department of EECS}{Fall
        2008} 
      % Developed and graded labs assignments where students build an
      % exokernel-style OS. Held weekly office hours to help students
      % with labs and OS fundamentals like virtual memory management,
      % interrupt handlers and process management.
      \vspace{-0.5em}
    \end{rentry}
    \begin{rentry}{Intro. to Digital Systems Lab (6.111)}{}
      \rline{Teaching Assistant, MIT Department of EECS}{Spring
        2008} 
      % Taught weekly recitations and helped students with labs using
      % FPGAs and Verilog. Helped students design and implement complex
      % final projects such as 3D object tracking. Received student
      % evaluation of 6.3/7.0, one of the highest ratings in the last 5
      % years.
      \vspace{-0.5em}
    \end{rentry}
    \begin{rentry}{Computation Structures (6.004)}{}
      \rline{Lab Assistant, MIT Department of EECS}{Spring 2007}
% Held office
%       hours to help students design and build a processor and small OS
%       kernel in simulation.
      \vspace{-0.5em}
    \end{rentry}
    \begin{rentry}{Intro. to Computer Science and
        Programming (6.00)}{}
      \rline{Lab Assistant, MIT Department of EECS}{Fall 2006}
 % Taught students
 %      basic computer science concepts such as recursion, abstraction
 %      and OOP.
      \vspace{-0.5em}
    \end{rentry}
  \end{rsection}
\end{rtable}

\begin{rtable}
  \vspace{1.0em}
  \begin{rsection}{Work\\Experience}
    \begin{rentry}{VMware, Inc.}{Cambridge, MA}      
      \rline{MTS, Virtual Machine Monitor
        Group}{Jan 2010 - Feb 2013}
      \vspace{-.5em}
    \end{rentry}
    \begin{rentry}{VMware, Inc.}{Cambridge, MA}      
      \rline{R\&D Intern, Virtual Machine Monitor Group}{Jul - Dec 2009}
      \vspace{-.5em}
    \end{rentry}
    \begin{rentry}{VMware, Inc.}{Cambridge, MA}      
      \rline{R\&D Intern, Core Performance Group}{Jun - Aug 2008}
      \vspace{-.5em}
    \end{rentry}
    \begin{rentry}{Quickware Engineering and Design}{Waltham, MA}
      \rline{Engineering Intern}{Jun - Aug 2007} 
      \vspace{-.5em}
    \end{rentry}
    \begin{rentry}{Cummins, Inc.}{Columbus, IN}
      \rline{Engineering Intern, Analysis Led Design}{Jun - Aug 2005}
      \vspace{-.5em}
    \end{rentry}
    \begin{rentry}{Cummins, Inc.}{Beijing, China}
      \rline{International Business Intern}{Jun - Jul 2004} 
      \vspace{-.5em}
    \end{rentry}
    \begin{rentry}{ArvinMeritor, Inc.}{Columbus, IN}
      \rline{Web Development Intern}{Aug 2003 - May 2004}
      \vspace{-.5em}
    \end{rentry}
  \end{rsection}
\end{rtable}
\end{document}
