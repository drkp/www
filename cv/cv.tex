% -*- mode: latex; TeX-PDF-mode:t; -*-
\documentclass[10pt,times]{report}
\usepackage[letterpaper,text={6.5in,9.5in},centering,nofoot]{geometry}
% use the showframe option in the above line to show the frame
\usepackage{microtype}


\setlength{\tabcolsep}{0pt}
\setlength{\parindent}{0pt}
\setlength{\parsep}{0pt}


% Separation between header and body
\newlength{\partgap}
\setlength{\partgap}{.2in}

% Additional separation between sections
\newlength{\sectiongap}
\setlength{\sectiongap}{.6em}

% Separation between entries
\newlength{\entrygap}
\setlength{\entrygap}{.25em}

\newlength{\sectioncolwidth}
\setlength{\sectioncolwidth}{1in}
\newlength{\colgap}
\setlength{\colgap}{.5em}
\newlength{\stuffwidth}
\setlength{\stuffwidth}{\textwidth}
\addtolength{\stuffwidth}{-\colgap}
\addtolength{\stuffwidth}{-\sectioncolwidth}
\addtolength{\stuffwidth}{-.5em}  % minipage margins?

\pagestyle{empty}


% From TeX by Topic
\def\ifEqString#1#2{\def\testa{#1}\def\testb{#2}%
  \ifx\testa\testb}

\newenvironment{rtable}{
  \begin{minipage}{\textwidth}
  }{
  \end{minipage}
}

\newenvironment{rentry}[3][xxx]{
  \begin{minipage}[t]{\hsize}
    \textbf{#2}\ifEqString{#1}{xxx}\relax\else, \textit{#1}\fi
    \hspace{\stretch{1}} #3 \\
  }{
    \removelastskip
  \end{minipage}
  \\[\entrygap]  % Useful for page squeezing
}

\newcommand{\rline}[2]{
  \begin{minipage}[t]{\hsize}
    #1 \hspace{\stretch{1}} #2
  \end{minipage} \\
}

\newenvironment{rsection}[1]{
  \begin{minipage}[t]{\sectioncolwidth}
    \textsc{#1}
  \end{minipage}
  \hspace{\colgap}
  \begin{minipage}[t]{\stuffwidth}
  }{
    \removelastskip
  \end{minipage}
  \\[\sectiongap]
}

\newenvironment{ritemize}{%
  \begin{list}{$\cdot$}{\topsep 0pt \parskip 0pt \partopsep 0pt
      \itemsep 0pt \parsep 0pt}%
}{\end{list}}

%%%%%%%%%%%%%%%%%%%%%%%%%%%%%%%%%%%%%%%%%%%%%%%%%%%%%%%%%%%%%%%%%%

\begin{document}

% Name
\begin{center}
  \LARGE{\sc{Irene Y. Zhang}}
\end{center}
\vspace{2mm}

% Contact info
\begin{tabular*}{\textwidth}{l@{\extracolsep{\fill}}r}
  185 NE Stevens Way & \textit{iyzhang@cs.washington.edu} \\
  Seattle, WA  98195 & \\ 
\end{tabular*}

\vspace{\partgap}

% Table
\begin{rtable}
  \begin{rsection}{Education}
    \begin{rentry}{University of Washington}{Seattle, WA}
      \rline{Ph.D. in Computer Science and Engineering}{}
      Advisors: Hank Levy and Arvind Krishnamurthy
    \end{rentry}

    \begin{rentry}{University of Washington}{Seattle, WA}
      \rline{M.S. in Computer Science and Engineering}{December 2013}
      Advisors: Hank Levy, Arvind Krishnamurthy, and Steve Gribble\\
      Thesis: \textit{Simplifying Mobile/Cloud Applications with Sapphire}
    \end{rentry}

    \begin{rentry}{Massachusetts Institute of Technology}{Cambridge,
        MA} \rline{M.Eng. in Electrical Engineering and Computer
        Science}{June 2009} Advisor: M. Frans Kaashoek\\
      Thesis: \textit{Efficient File Distribution in a Flexible, Wide-area
        File System}%\\Graduate GPA: 5.0/5.0
    \end{rentry}

    \begin{rentry}{Massachusetts Institute of Technology}{Cambridge, MA}        
        \rline{S.B. in Computer Science and Engineering}{June 2008}
        %Undergraduate GPA: 4.7/5.0
    \end{rentry}
  \end{rsection}

  \begin{rsection}{Interests}
    Operating systems, virtualization, distributed systems and networking
  \end{rsection}
  
  \begin{rsection}{Research}
    \begin{rentry}{Optimistic Replicated Two-Phase Commit}{}
      Optimistic Replicated Two-Phase Commit (OR-2PC) is the first
      \emph{co-designed} protocol that integrates protocols for
      consistent replication, atomic commitment and concurrency
      control to take advantage of cross-protocol
      optimizations. OR-2PC uses a new optimistic concurrency control
      protocol that enforces consistency across replicas as well as
      across transactions with a new ordering technique based on
      loosely synchronized clocks. OR-2PC commits transactions in a
      single round-trip and eliminates bottlenecks associated with
      leader-based Paxos, while maintaining the same consistency
      guarantees, general-purpose transaction model and replication
      requirements as standard protocols.
    \end{rentry}

    \begin{rentry}{Customizable and Extensible Deployment for
        Mobile/Cloud Applications}{}
      Sapphire is a new distributed programming platform that provides
      \emph{customizable} and \emph{extensible} deployment of
      mobile/cloud applications. Rather than directly deploying the
      application, application programmers use a library of deployment
      managers, which customize the platform for their
      application. Further, Sapphire has an interface for developing
      new deployment managers, allowing programmers to extend the
      platform. This flexibility enables programmers to separate
      deployment logic from their application, while maintaining
      fine-grained control over performance trade-offs.
    \end{rentry}

    \begin{rentry}{Arrakis}{}
      Arrakis is a new operating system that provides high performance
      I/O by taking advantage of hardware virtualization
      technology. Hardware virtualization technologies are designed to
      eliminate the hypervisor from fast-path I/O operations. Arrakis
      takes this technology a step further by using it to eliminate
      the operating system as well, allowing applications to directly
      access the hardware during normal execution and providing
      significantly better performance, reliability and
      customizability.
    \end{rentry}

    \begin{rentry}{Improving VM Checkpoint Restore Performance}{} 
      With collaborators at VMware, I developed two techniques for
      improving the performance of restoring checkpointed virtual
      machines. The first estimates and prefetches the working set of
      the checkpointed VM on restore, improving the responsiveness of
      the VM during restore. The second groups memory pages together
      on disk that are likely to be accessed together, improving disk
      efficiency during restore.
    \end{rentry}

    % \begin{rentry}{Transactional Consistency and Automatic Cache Management}{}
    %   In-memory application caches like memcached introduce
    %   significant complexity to web applications because they provide
    %   a simple get-put interface, violate the consistency guarantees
    %   of the underlying database and leave cache invalidations to the
    %   application. TxCache addresses these problems by providing a
    %   programming model for caching, transactional consistency across
    %   the entire system and automatic cache invalidations generated by
    %   the database.
    % \end{rentry}

    % \begin{rentry}{WheelFS and Large File Distribution}{}
    %   The goal of WheelFS is to provide a general wide-area storage
    %   solution with a standard POSIX interface. The challenge is that
    %   any storage system operating in the wide-area must make
    %   trade-offs for performance and trade-off decisions are often
    %   better made by applications. WheelFS solves this problem by
    %   allowing applications to configure the file system using
    %   keywords embedded in the pathname. My thesis work focused on
    %   adding support for efficient file distribution to WheelFS for
    %   distributing large files and handling flash crowds.
    % \end{rentry}        
  \end{rsection}
\end{rtable}

\begin{rtable}
  %\vspace{1.0em}
%   \begin{rsection}{}
%     \begin{rentry}{Fresh Breeze Multiprocessor Architecture}{}
%       Work with Prof. Jack Dennis implementing a cycle-accurate
%       simulator for the Fresh Breeze multiprocessor architecture.  Led
%       a 3 student team in the design and implementation of a tool for
%       interacting with the Fresh Breeze simulator. The goal of the
%       Fresh Breeze architecture is to provide a sound base for
%       executing programs written following the principles of modular
%       software construction. The architecture introduces three major
%       departures from conventional multiprocessor design: simultaneous
%       multithreading, global shared address space, and no memory
%       update, cycle-free heap.
%     \end{rentry}
%  \end{rsection}
\vspace{1.0em}
 \begin{rsection}{Publications}
   I. Zhang, N. K. Sharma, A. Szekeres, D. R. K. Ports,
   A. Krishnamurthy. \textit{Optimistic, Replicated Two-Phase Commit.}
   In submission.\\\vspace{-0.5em}

   I. Zhang, A. Szekeres, D. Van Aken, I. Ackerman, S. D. Gribble,
   A. Krishnamurthy, H. Levy. \textit{Customizable and Extensible
     Deployment for Mobile/Cloud Applications.} In
   submission.\\\vspace{-0.5em}

   S. Peter, J. Li, D. Woos, I. Zhang, D. R. K. Ports, A. Krishnamurthy,
   T. Anderson, T. Roscoe. \textit{Arrakis: The Operating System is
     the Control Plane.} In submission.\\\vspace{-0.5em}

   S. Peter, J. Li, D. Woos, I. Zhang, D. R. K. Ports, T. Anderson,
   A. Krishnamurthy, M. Zbikowski. \textit{Towards High-Performance
     Application-Level Storage Management.} In Proc. of HotStorage
   '14. \\\vspace{-0.5em}

    I. Zhang, T. Denniston, Y. Baskakov, A. Garthwaite. \textit{Optimizing
      VM Checkpointing for Restore Performance in VMware ESXi.} In
    Proc. of USENIX ATC '13. San Jose, CA. June 2013.\\\vspace{-0.5em}

    I. Zhang, A. Garthwaite, Y. Baskakov, K. C. Barr. \textit{Fast
      Restore of Checkpointed Memory Using Working Set
      Estimation.} In Proc. of VEE '11. Newport Beach, CA. March
    2011.\\\vspace{-0.5em}

    D. R. K. Ports, A. Clements, I. Zhang, S. Madden,
    B. Liskov. \textit{Transactional Consistency and Automatic
      Management in an Application Data Cache.} In Proc. of OSDI
    '10. Vancouver, Canada. October 2010.\\\vspace{-0.5em}

    J. Stribling, Y. Sovran, I. Zhang, X. Pretzer, J. Li,
    M. F. Kaashoek, R. Morris. \textit{Flexible, Wide-Area Storage for
      Distributed Systems with WheelFS.} In Proc. of NSDI '09. Boston,
    MA. April 2009.
 \end{rsection}

  \begin{rsection}{Posters \&\\Extended\\Abstracts}
   I. Zhang, N. K. Sharma, A. Szekeres, D. R. K. Ports,
   A. Krishnamurthy. \textit{Optimistic, Replicated Two-Phase Commit.}
   Poster at APSys '14. Beijing, China. June 2014.\\\vspace{-0.5em}

    I. Zhang, A. Szekeres, D. Van Aken, I. Ackerman, S. D. Gribble,
   A. Krishnamurthy, H. Levy. \textit{Customizable and Extensible
     Deployment for Mobile/Cloud Applications.} Work-in-Progress talk
   at SOSP '13. Farmington, PA. November 2013.\\\vspace{-0.5em}

    I. Zhang, A. Garthwaite, Y. Baskakov, K. C. Barr, J. Pool,
    K. Christopher. \textit{Fast Restore of Checkpointed Memory Using
      Working Set Estimation.} Poster at SOSP '09. Big Sky,
    MT. October 2009.\\\vspace{-0.5em} 

   I. Zhang, K. C. Barr. \textit{Improving VMware Workstation Restore
      using Working Set Estimation.} Poster at VMworld '08. Las Vegas,
    NV. September 2008.
  \end{rsection}

  \vspace{1.0em}
  \begin{rsection}{Patents}
    US Patent App. 12/559,484.\\
    Saving and Restoring State Information for Virtualized Computer Systems.\\
    I. Zhang, K. C. Barr, G. Venkitachalam, I. Ahmad, A. Garthwaite, J. Pool.\\\vspace{-0.5em}
    
  \end{rsection}

  % \begin{rsection}{Talks}
  %   Saving and Restoring State Information for Virtualized Computer Systems.
  %   I. Zhang, K. C. Barr, G. Venkitachalam, I. Ahmad, A. Garthwaite, J. Pool.
  %   US Patent App. 12/559,484.
  % \end{rsection}

\end{rtable}

\begin{rtable}
  \vspace{1.0em}
  \begin{rsection}{Work\\Experience}
    \begin{rentry}{VMware, Inc.}{Cambridge, MA}      
      \rline{MTS, Virtual Machine Monitor
        Group}{Jan 2010 - Feb 2013}
      \vspace{-.5em}
    \end{rentry}
    \begin{rentry}{VMware, Inc.}{Cambridge, MA}      
      \rline{R\&D Intern, Virtual Machine Monitor Group}{Jul - Dec 2009}
      \vspace{-.5em}
    \end{rentry}
    \begin{rentry}{VMware, Inc.}{Cambridge, MA}      
      \rline{R\&D Intern, Core Performance Group}{Jun - Aug 2008}
      \vspace{-.5em}
    \end{rentry}
    \begin{rentry}{Quickware Engineering and Design}{Waltham, MA}
      \rline{Engineering Intern}{Jun - Aug 2007} 
      \vspace{-.5em}
    \end{rentry}
    \begin{rentry}{Cummins, Inc.}{Columbus, IN}
      \rline{Engineering Intern, Analysis Led Design}{Jun - Aug 2005}
      \vspace{-.5em}
    \end{rentry}
    \begin{rentry}{Cummins, Inc.}{Beijing, China}
      \rline{International Business Intern}{Jun - Jul 2004} 
      \vspace{-.5em}
    \end{rentry}
    \begin{rentry}{ArvinMeritor, Inc.}{Columbus, IN}
      \rline{Web Development Intern}{Aug 2003 - May 2004}
      \vspace{-.5em}
    \end{rentry}
  \end{rsection}

  \vspace{1.0em}
  \begin{rsection}{Teaching}
    \begin{rentry}{Operating Systems Engineering (6.828)}{Sept - Dec
        2008}
      \rline{Teaching Assistant, MIT Department of EECS}{} 
      Developed and graded labs assignments where students build an
      exokernel-style OS. Held weekly office hours to help students
      with labs and OS fundamentals like virtual memory management,
      interrupt handlers and process management.
    \end{rentry}    

    \begin{rentry}{Intro. to Digital Systems Lab (6.111)}{Jan - May
        2008}
      \rline{Teaching Assistant, MIT Department of EECS}{} 
      Taught weekly recitations and helped students with labs using
      FPGAs and Verilog. Helped students design and implement complex
      final projects such as 3D object tracking. Received student
      evaluation of 6.3/7.0, one of the highest ratings in the last 5
      years.
    \end{rentry}    

    \begin{rentry}{Computation Structures (6.004)}{Jan - Dec 2007}
      \rline{Lab Assistant, MIT Department of EECS}{} 
      Held office hours to help students design and build a processor
      and small OS kernel in simulation.
    \end{rentry}

    \begin{rentry}{Intro. to Computer Science and
        Programming (6.00)}{Sept - Dec 2006}
      \rline{Lab Assistant, MIT Department of EECS}{} 
      Taught students basic computer science concepts using Python
      such as recursion, abstraction and OOP.
    \end{rentry}
  \end{rsection}

  \vspace{1.0em}
  \begin{rsection}{Service}
    \begin{rentry}{UW Visit Days 2014}{}
      Organized and ran the Visit Days for PhD students in the
      Computer Science and Engineering Department.
    \end{rentry}    

    \begin{rentry}{Women's Outreach at VMware}{}
      Organized and led events to recruit and support women in
      computer science programs around the country.
    \end{rentry}    
  \end{rsection}
\end{rtable}

\begin{rtable}
  \vspace{1.0em}
  \begin{rsection}{Honors and\\Awards}
    \begin{rentry}{National Science Board Annual Board Meeting Student
        Panel}{2013}
    \end{rentry} \vspace{-0.5em}

    \begin{rentry}{National Science Foundation Fellowship}{2013}
    \end{rentry} \vspace{-0.5em}

    \begin{rentry}{ARCS Foundation Fellowship}{2012}
    \end{rentry} \vspace{-0.5em}

    \begin{rentry}{Jeff Dean and Heidi Hopper Endowed Regental Fellowship}{2012}
    \end{rentry} \vspace{-0.5em}

    \begin{rentry}{VMware Academic Program Top Intern Project}{2008}
    \end{rentry} \vspace{-0.5em}
    
    \begin{rentry}{CRA Outstanding Undergraduate Award, Honorable
      Mention}{2008}
    \end{rentry} \vspace{-0.5em}

    \begin{rentry}{Officer, Eta Kappa Nu EECS Honor Society}{2008}
    \end{rentry} \vspace{-0.5em}

    \begin{rentry}{Northern Telecom/BNR Award for Best Undergrad. Lab Project}{2006}
    \end{rentry}
  \end{rsection}
  \vspace{-1em}  
\end{rtable}
\end{document}
